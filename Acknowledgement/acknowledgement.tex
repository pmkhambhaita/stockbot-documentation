% ************************** Thesis Acknowledgements **************************

\begin{preface}      


This project represents many months of dedication, learning, and passion for creating technology that solves real-world problems. As the son of a hard-working warehouse employee at Currys, I've witnessed first-hand the challenges faced by retail workers who navigate large stockrooms daily. Their efficiency directly impacts customer satisfaction, yet the tools available to them often fall short of modern technological capabilities.

\textbf{}
\newline
StockBot was born from these observations: a solution designed to optimise the simple but critically important task of item collection in warehouse environments. The inefficiencies I observed during visits to my father's workplace provided inspiration for the solution developed here.

\textbf{}
\newline
This application has been proudly developed using open-source tools on Fedora, the leading edge of secure, accessible and transparent technology. From the Python language to the SQLite database system, every component of StockBot is built upon FOSS and using FOSS tools where possible.

\textbf{}
\newline
This project demonstrates that even complex computational challenges can be addressed with relatively simple, elegant solutions when we take the time to thoroughly understand the problem space. The pathfinding algorithms employed here may not be revolutionary, but their application to the specific constraints of retail warehousing creates a tool that can meaningfully improve workplace efficiency and reduce physical strain for warehouse workers.

\textbf{}
\newline
I hope that StockBot serves as both a practical solution and an example of how computer science students 
from all walks of life can develop applications that address real-world challenges. Thank you for taking the time to read my project, and I hope you find it interesting \& insightful.

\textbf{}
\newline
— Praveet Minash Khambhaita \newline
  \href{https://github.com/pmkhambhaita}{@pmkhambhaita}


\end{preface}
